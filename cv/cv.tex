\documentclass[11pt,a4paper]{moderncv}
\usepackage[utf8]{inputenc}
\usepackage[francais]{babel}

\usepackage[top=1.1cm, bottom=1.1cm, left=1.5cm, right=1.5cm]{geometry}


\moderncvtheme[orange]{classic}

\firstname {Marc-André}
\lastname {Allard}
\address{4634 Avenue Dupuis Apt. 2}{Montréal, Québec  H3W 1N3}
\email {m.a.allard10@gmail.com}
\mobile {(438) 888-1807}
\homepage {mcaead10.github.io}
\social[github]{mcaead10}
\title{Développeur logiciel}
\extrainfo{Bilingue: français et anglais}

\begin{document}
   \makecvtitle
   \section{Formations}
   \cventry{2013--2016}{Université du Québec à Montréal}{Baccalauréat en Informatique et Génie Logiciel}{}{}{}
   \cventry{2012--2013}{Université du Québec à Montréal}{Certificat en Sciences Comptables}{}{}{}
   \cventry{2008--2011}{CÉGEP de l'Abitibi-Témiscamingue}{Techniques de Comptabilité et de Gestion}{}{}{}
   \section{Experiences Professionnelles}
   \cventry{Septembre 2015 \\ à Août 2016}{Stagiaire en Informatique}{Morneau Shepell}{Montréal}{}{Développement et création de l'architecture d'une interface de programmation en C\# pour l'application Ranorex, pour permettre à l'équipe d’assurance qualité d'automatiser leurs scénarios de test.}
   \section{Réalisations}
   \cventry{Application Web}{FoodTruck Locator}{Java, PostgreSQL, JQuery, Bootstrap, Spring Boot}{}{}{Permet l'affichage des camions restaurants de la ville de Montréal sur une carte OpenStreetMap avec la bibliothèque Leaflet. Les données de localisation des camions restaurants sont récoltées grâce au Portail de Données Ouvertes de la ville de Montréal.}
   \cventry{Application Web}{EZBudget}{Java, MySQL, AngularJS, Bootstrap, SpringBoot}{}{}{Permet à plusieurs utilisateurs de suivre leurs activités financières personnelles. L'application aide à comptabiliser ses dépenses et ses entrées d'argent puis affiche un somaire accompagné de graphiques.}
   \cventry{Ligne de commande}{Path Finding}{C++}{}{}{Réalisée dans le cadre d'un cours de structures de données. Implémente l’algorithme de Dijkstra. Les objets, graphes et coordonnées géographiques ont été créés avec la librairie standard C++.}
   \section{Compétences en Informatique}
   \cvitem{Langages}{Java, Python, C\#, C++, C, Haskell}
   \cvitem{Technologies web}{HTML5, CSS3, JavaScript, Jquery}
   \cvitem{Bases de données}{MySQL, PostgreSQL, Oracle}
   \cvitem{Systèmes}{Linux (Ubuntu, Mint, Debian), Mac OS, Windows}
   \cvitem{Logiciels}{Suite JetBrains, Eclipse, Git, VIM, Visual Studio}
   \cvitem{Autres}{Développement Agile, GoF, GRASP, UML, REST, Spring, Expression régulière, Bash, Maven, Gradle, JSON, XML}
   \section{Implications Sociales}
   \cventry{2016--2017}{Président}{Association Générale des Étudiantes et Étudiants en Informatique}{UQÀM}{}{}
   \cventry{2014--2016}{Trésorier}{Association Générale des Étudiantes et Étudiants en Informatique}{UQÀM}{}{}
   \cventry{2016}{VP-Finance}{CS GAMES - Comité Organisateur des Sciences Informatique}{UQÀM}{}{}
   \cventry{2017}{Bénévole}{CS GAMES}{ÉTS}{}{}
   \section{Centres d'intérêts}
   \cvitem{Domotique}{Prototypage et installation avec des Raspberry Pi afin d'automatiser et de contrôler des appareils électriques.}
\end{document}
