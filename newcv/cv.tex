\documentclass[11pt,a4paper]{moderncv}
\usepackage[utf8]{inputenc}
\usepackage[francais]{babel}

\usepackage[top=1.1cm, bottom=1.1cm, left=1.8cm, right=1.8cm]{geometry}


\moderncvtheme[orange]{classic}

\firstname {Marc-André}
\lastname {Allard}
\address{4634 Avenue Dupuis Apt. 2}{Montréal, Québec  H3W 1N3}
\email {m.a.allard10@gmail.com}
\mobile {(438) 888-1807}
\homepage {mcaead10.github.io}
\social[github]{mcaead10}
\title{Développeur logiciel junior}

\begin{document}
   \makecvtitle
   \section{Formations}
   \cventry{2013--2016}{UQÀM}{Baccalauréat en Informatique et Génie Logiciel}{}{}{}
   \cventry{2012--2013}{UQÀM}{Certificat en Sciences Comptables}{}{}{}
   \cventry{2008--2011}{CÉGEP-AT}{Techniques de Comptabilité et de Gestion}{}{}{}
   \section{Experiences Professionnelles}
   \cventry{Septembre 2015 \\ à Août 2016}{Stagiaire en Informatique}{Morneau Shepell}{Montréal}{}{Développement et création de l'architecture d'un interface de programmation en C\# pour l'application Ranorex, pour permettre à l'équipe d'assurance qualité de créer l'automatisation de leurs scénarios de test.}
   \section{Réalisations}
   \cventry{Application Web}{FoodTurck Locator}{Java, PostgreSQL, JQuery, Bootstrap, Spring Boot}{}{}{Cette application permet l'affichage des camions restaurents de la ville de Montréal sur une carte OpenStreetMap avec la bibliothèque Leaflet. Les données de localisation des camions restaurents sont récoltées grâce à l'Opendata de la ville de Montréal.}
   \cventry{Application Web}{EZBudget}{Java, MySQL, AngularJS, Bootstrap, SpringBoot}{}{}{Cette application permet à plusieurs utilisateurs de faire un suivi de ses activités financières presonnelles. L'application permet de catégoriser ses dépenses et ses entrés d'argent afin d'afficher des résumés graphiques.}
   \cventry{Ligne de commande}{Path Finding}{C++, STD}{}{}{Cette application implémente l'algorithme de Dijkstra. Les objets graph et coordonnées géographique ont été créés avec la STD.}
   \section{Compétences en Informatique}
   \cvitem{Langages}{Java, Python, C\#, C++, C, Haskell}
   \cvitem{Langages web}{HTML5, CSS3, JavaScript, Jquery}
   \cvitem{Langages autres}{Bash, Latex}
   \cvitem{Bases de données}{MySQL, PostgreSQL, Oracle}
   \cvitem{Systèmes}{Linux (Ubuntu, Mint, Debian), Mac OS, Windows}
   \cvitem{Logiciels}{Suite JetBrains, Eclipse, Git, VIM, Visual Studio}
   \cvitem{Autres}{Programmation Agile, GoF, GRASP, UML, REST, Spring, Expression Régulière, Maven, Gradle, JSON, XML}
   \section{Implications Sociales}
   \cventry{2016--2017}{Président}{Association Générale des Étudiantes et Étudiants en Informatique}{UQÀM}{}{}
   \cventry{2014--2016}{Trésorier}{Association Générale des Étudiantes et Étudiants en Informatique}{UQÀM}{}{}
   \cventry{Édition 2016}{VP-Finance}{CS GAMES - Comité Organisateur des Sciences Informatique}{UQÀM}{}{}
   \section{Centres d'intérêts}
   \cvitem{Domotique}{Prototypage et installation avec des RaspberryPi afin d'automatiser et de contrôler des appareils électriques.}
\end{document}
